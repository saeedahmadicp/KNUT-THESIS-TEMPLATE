\newcommand{\thesisTitleKorean}{Cross-Modal 지식 증류와 Hybrid CNN-Transformer를 활용한 향상된 의료 이미지 분할 기법}

{\centering \Large \bf \thesisTitleKorean \par} \vspace{1.0cm}

의료 영상 분할은 의료 영상 데이터 내의 해부학적 구조, 기관 및 이상을 정확하게 묘사하는 중요한 작업으로, 진단, 치료 계획 및 다양한 질병의 정량적 분석에 근본적인 역할을 한다. 이러한 이미지를 수동으로 분석하려면 방사선 전문의의 높은 전문 지식이 필요하지만, 이 방법은 인간의 오류에 취약하고 확장성에 한계가 있다. 따라서 의료 영상의 자동 분석에 대한 요구가 커지고 있으며, 이는 분석 과정을 가속화하고 접근성과 효율성을 증대시킬 수 있다. 본 연구는 효율적인 자동 뇌종양 분할 알고리즘 개발에 중점을 둔다. 이를 위해 다중 모드 뇌종양 분할을 위한 구성 및 최적화된 하이브리드 잔여 주의 UNet(COHRA-UNet)과 단일 모드 분할을 위한 새로운 다중 교사 교차 모달 지식 증류(MTCM-KD) 프레임워크를 제안한다.

첫 번째 연구에서는 활성화 함수, 정규화 기술, 업샘플링 방법, 손실 함수 등 다양한 하이퍼파라미터를 종합적으로 조사하였다. 또한, 공간, 채널 및 트랜스포머 기반 self-attention을 포함한 다양한 주의 메커니즘의 영향을 평가하고, UNet 아키텍처의 뇌종양 분할 성능을 향상시키기 위한 최적의 설정을 탐색하였다. 이 과정에서 트랜스포머 기반 CNN 아키텍처의 다양한 하이브리드화를 검토하였다. 제안된 COHRA-UNet은 이러한 최적의 매개변수와 구성을 통합하여 우수한 성능을 발휘하며, 기존의 최신 방법론들을 능가하는 결과를 보였다.

두 번째 연구에서는 임상 환경에서 사용 가능한 MRI 양식이 제한적인 문제를 해결하기 위해 MTCM-KD 프레임워크를 개발하였다. 이 프레임워크는 임상에서 자주 사용되는 T\textsubscript{1ce} MRI 시퀀스를 효과적으로 분할할 수 있도록 설계되었으며, T\textsubscript{1} MRI 스캔과 구조적 유사성을 공유하여 충분한 정보를 제공한다. 프레임워크는 성과 중심 응답 기반 KD와 협력적 심층 감독 융합 학습(CDSFL)이라는 두 가지 고유한 지식 증류(KD) 전략을 통합한다. 성과 중심 응답 기반 KD 전략은 각 교사 모델의 성과에 따라 신뢰도 가중치를 조정하여 KD 프로세스를 성과에 맞게 조정한다. 또한, CDSFL 모듈은 상호 학습을 촉진하여 다중 교사 모델의 학습 능력을 강화한다. 이러한 모델들의 결합된 지식은 학생 모델에 증류되어 딥러닝 감독을 향상시킨다. BraTS 데이터 세트를 통한 포괄적인 테스트 결과, 우리 프레임워크는 T\textsubscript{1ce} 및 T\textsubscript{1} 양식의 단일 모드 분할에서 기존의 최신 기술을 능가하는 유망한 결과를 보였다.